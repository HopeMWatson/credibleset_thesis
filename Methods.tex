## Why use Simulations?
Simulations are the gold standard in evaluating methodologies relevant to causal variants. In the "real-world", that is genotyping for actual phenotypes of interest (typically a disease), we do not know the causal variant. It is not possible evaluate both real-world data and the robustness of the methodology simulataneously. This is because the lack of certainty and variability of both parameters may be the reason for the outcomes; a basic concept of the scientific method -- test one variable at a time and hold all other variables the same as controls. 

## Running Simulations

Simulations were run by through a series of steps, leveraging relevant available R packages. Full code described here is available on: https://github.com/HopeMWatson/credibleset_thesis/blob/master/coverage%20.Rmd

The libraries utilised in this analysis were devtools to load the simGWAS and coloc packages directly from github. ggplot2 package was used to produce figures. 

### simGWAS
The simGWAS package directly simulates GWAS summary data, without indivudal data as an intermediate step. The expected statistics are mathemathically derived for any set of causal variants and their effects sizes, conditional upon control haplotype frequencies [@Fortune2018]. The arguements for simGWAS to run require a specifation of the causal SNP, its effect size (OR), and allele (MAF) and haplotypes frequencies. simGWAS then produce simulated z-scores for each SNP. A list of z-scores was created for each SNP by running the simulation 100 times. The final output from simGWAS was a 100x100 matrix, which contained data on 100 SNPs with 100 z-scores. The z-scores were turned into p-values. 

### Finemap.abf 
From the assigned p-values, posterior probailites are calculated. This process was outlined above, showing how approximate bayes factors (ABF) are created from the observed data and the uniformative prior. The finemap.abf function works on summary statistic data, inputs as either 1) p-values, as used here or 2) coefficients and coefficient variance. Other functions in the coloc package can be utilised for full genotyped data with different statistical approaches. 

The finemap.abf function works by specifying input arguements for 1) p-values 2) sample size 3) MAF 4) ratio of cases to controls (s), and 5) type =cc. The prior is set to p=1e-04 as a default for an uniformative prior. An empty matrix was created in to map each to place each posterior probability into the list back into the dataframe. The finemap.abf function was then looped over each SNP and its respective list of p-values to create posterior probabilities. The final output from finemap.abf was a 100x100 matrix, which contained data on 100 SNPs with 100 respective posterior probabilites of the SNP being the specified causal variant. 

### Credset 